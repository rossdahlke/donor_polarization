\documentclass[12pt,]{article}
\usepackage[left=1in,top=1in,right=1in,bottom=1in]{geometry}
\newcommand*{\authorfont}{\fontfamily{phv}\selectfont}
\usepackage[]{mathpazo}


  \usepackage[T1]{fontenc}
  \usepackage[utf8]{inputenc}




\usepackage{abstract}
\renewcommand{\abstractname}{}    % clear the title
\renewcommand{\absnamepos}{empty} % originally center

\renewenvironment{abstract}
 {{%
    \setlength{\leftmargin}{0mm}
    \setlength{\rightmargin}{\leftmargin}%
  }%
  \relax}
 {\endlist}

\makeatletter
\def\@maketitle{%
  \newpage
%  \null
%  \vskip 2em%
%  \begin{center}%
  \let \footnote \thanks
    {\fontsize{18}{20}\selectfont\raggedright  \setlength{\parindent}{0pt} \@title \par}%
}
%\fi
\makeatother




\setcounter{secnumdepth}{0}

\usepackage{longtable,booktabs}



\title{Political Donor Polarization: Observing Consumptive Behavior using a
Network Approach \thanks{Code and data available at: github.com/rossdahlke}  }



\author{\Large Ross Dahlke\vspace{0.05in} \newline\normalsize\emph{}  }


\date{}

\usepackage{titlesec}

\titleformat*{\section}{\normalsize\bfseries}
\titleformat*{\subsection}{\normalsize\itshape}
\titleformat*{\subsubsection}{\normalsize\itshape}
\titleformat*{\paragraph}{\normalsize\itshape}
\titleformat*{\subparagraph}{\normalsize\itshape}





\newtheorem{hypothesis}{Hypothesis}
\usepackage{setspace}


% set default figure placement to htbp
\makeatletter
\def\fps@figure{htbp}
\makeatother

\usepackage{graphicx}

% move the hyperref stuff down here, after header-includes, to allow for - \usepackage{hyperref}

\makeatletter
\@ifpackageloaded{hyperref}{}{%
\ifxetex
  \PassOptionsToPackage{hyphens}{url}\usepackage[setpagesize=false, % page size defined by xetex
              unicode=false, % unicode breaks when used with xetex
              xetex]{hyperref}
\else
  \PassOptionsToPackage{hyphens}{url}\usepackage[draft,unicode=true]{hyperref}
\fi
}

\@ifpackageloaded{color}{
    \PassOptionsToPackage{usenames,dvipsnames}{color}
}{%
    \usepackage[usenames,dvipsnames]{color}
}
\makeatother
\hypersetup{breaklinks=true,
            bookmarks=true,
            pdfauthor={Ross Dahlke ()},
             pdfkeywords = {polarization, political donations, network analysis, state politics},  
            pdftitle={Political Donor Polarization: Observing Consumptive Behavior using a
Network Approach},
            colorlinks=true,
            citecolor=blue,
            urlcolor=blue,
            linkcolor=magenta,
            pdfborder={0 0 0}}
\urlstyle{same}  % don't use monospace font for urls

% Add an option for endnotes. -----


% add tightlist ----------
\providecommand{\tightlist}{%
\setlength{\itemsep}{0pt}\setlength{\parskip}{0pt}}

% add some other packages ----------

% \usepackage{multicol}
% This should regulate where figures float
% See: https://tex.stackexchange.com/questions/2275/keeping-tables-figures-close-to-where-they-are-mentioned
\usepackage[section]{placeins}


\begin{document}
	
% \pagenumbering{arabic}% resets `page` counter to 1 
%
% \maketitle

{% \usefont{T1}{pnc}{m}{n}
\setlength{\parindent}{0pt}
\thispagestyle{plain}
{\fontsize{18}{20}\selectfont\raggedright 
\maketitle  % title \par  

}

{
   \vskip 13.5pt\relax \normalsize\fontsize{11}{12} 
\textbf{\authorfont Ross Dahlke} \hskip 15pt \emph{\small }   

}

}








\begin{abstract}

    \hbox{\vrule height .2pt width 39.14pc}

    \vskip 8.5pt % \small 

\noindent American politics has recently been defined by unprecedented levels of
partisan polarization. Given the concurrent rise of the amount of money
in politics, many have suggested a connection between money in politics
and polarization. This paper uses the occurence of a specific polarizing
event, former Wisconsin Governor Scott Walker's introduction and passage
of Act 10, to analyze the relationship between donor polarization and
mass polarization. Using political donation data from the Wisconsin
Campaign Finance Information System (CFIS) and using the network science
measure of modularity, this paper shows that political donor networks
polarized during the 2012 election cycle at the same time as the
electorate. This result suggests that political donors were likely not
the main contributors to the polarization in the state and provides
evidence for the `consumption' model of political donations.


\vskip 8.5pt \noindent \emph{Keywords}: polarization, political donations, network analysis, state politics \par

    \hbox{\vrule height .2pt width 39.14pc}



\end{abstract}


\vskip -8.5pt


 % removetitleabstract

\noindent \doublespacing 

Political campaign finance plays an important role in the American
political system. This significance is evidenced by the attention that
academic researchers pay to the topic as well as the many different
contexts in which campaign finance is studied. For example, research has
been conducted on the impact of political donations on roll-call voting
in the U.S. Congress (Roscoe and Jenkins 2005; Stratmann 1991), gender
representation in political parties (Crowder-Meyer and Cooperman 2018;
Barber, Butler, and Preece 2016; Kitchens and Swers 2016; Thomsen and
Swers 2017), ability to win political campaigns (Bonica 2017; Bonneau
2007), the connection between money raised and public attention (Ellis,
Ripberger, and Swearingen 2017), judicial function (Palmer and Levendis
2008), perceptions of corruption (Bowler and Donovan 2015), political
economy and stock returns (Akey 2015; Fowler, Garro, and Spenkuch 2020;
Cooper, Gulen, and Ovtchinnikov 2010), and the significant amount of
time that candidates and legislators devote to fundraising
(Torres-Spelliscy 2017).

Even though political donors are believed to play an out-sized role in
democracy, the psychological processes of donors is thought to be
similar to ordinary voters. Political donations can be thought of an
extension of voting. In other words, both actions are political
consumption that seek to improve a preferred candidate's chances of
winning. Ansolabehere, de Figueiredo and Snyder summarized this idea by
stating, ``In our view, campaign contributing should not be viewed as an
investment, but rather as a form of consumption--or, in the language of
politics, participation'' (2003). Donations can be seen as an outlet for
motivated citizens to increase their participation beyond just turning
out to vote when they perceive the stakes of elections to be high (Hill
and Huber 2017).

The folk-theory of political donors is of smokey backrooms and
access-oriented donors who seek to have a direct influence on policy
making. However, even when donors contribute to legislators that
maximize their economic interests, donations are not found to be
motivated by existing policy agreements and not an expectation of access
(Barber, Canes-Wrone, and Thrower 2016). Even donations from business
executives have been found to be ``best understood as purchases of `good
will' whose returns, while positive in expectation, are contingent and
rare'' (Gordon, Hafer, and Landa 2007).

Although the psychological process of making a political campaign
contribution can be thought of as similar to voting, there are
significant demographic and ideological differences between donors and
voters. People with lower incomes, less education, and do not work in
professional and managerial jobs are less likely to be politically
engaged, including making political donations (Laurison 2016). Donors to
the Democratic and Republican parties were summarized as being
``Limousine Liberals'' and ``Corporate Conservatives'' (Francia et al.
2005). In addition, while Democrats and Republicans draw their bases of
electoral support from different geographic bases, major campaign donors
are highly concentrated geographically. These ``big-donor
neighborhoods'' are unrepresentative of the country as a whole and point
to these communities having a distinct political culture (Bramlett,
Gimpel, and Lee 2011). In both parties, donors are more ideologically
extreme than non-donating voters (Hill and Huber 2017; Francia et al.
2003) and wealthy donors who make up the ``big money'' in politics are
especially partisan (McCarty, Poole, and Rosenthal 2006).

This ideological extremity shown by political donors has led some
scholars to suggest that political donors are contributors to the
partisan polarization of the politics of the United States (Francia et
al. 2005). This idea is supported by the observation that both political
polarization and campaign spending have risen in conjunction (McCarty,
Poole, and Rosenthal 2006). However, there little evidence for the
causal relationship of donors causing political polarization. Many have
found that political donations don't influence polarization (Harden and
Kirkland 2016; Raja and Wiltse 2012; Keena and Knight-Finley 2019).
Furthermore, it is likely the case that the causal arrow flows the other
direction, and it is a more polarized electorate and candidates that
have led to more polarized donors (Harden and Kirkland 2016; Raja and
Wiltse 2012; Keena and Knight-Finley 2019).

{[}Add in section about consequences of polarized donors / why is it
inmportant that we understand what is causing greater donor
polarization{]}

Studying polarization, particularly among political donors can be
difficult because of the myriad of potential confounding factors that
can contribute to polarization (Harden and Kirkland 2016). In addition,
polarization is generally a phenomenon that gradually increases or
decreases over time (Pew Research Center 2017). However, this paper
leverages a singular event, former Wisconsin Governor Scott Walker's
proposition and passage of Act 10, a ``budget repair bill'' that ended
collective bargaining for teachers unions, to examine political donor
polarization in the state of Wisconsin. Given the recent research which
has pointed to the polarization of political donors as being
\emph{reactive} instead of \emph{causal} to broader polarization, we
could expect political donors to follow the trend of voters and polarize
after introduction of Act 10 and subsequent events.

\textbf{\(H_{1}\): Political donors in the State of Wisconsin polarized
during the 2011-2012 election cycle compared to the 2009-2010 election
cycle and maintained their level of polarization in the 2013-2014
election cycle.}

If this hypothesis is supported, these results would strengthen the
evidence for politics donors being \emph{reactive} to their political
environment as we would expect under Ansolabehere, de Figueiredo and
Snyder's consumption model of political giving.

Alternatively, if political donors are instead \emph{contributors} to
mass and legislative polarization, as is suggested by some scholars, we
would expect to see hypothesis 2.

\textbf{\(H_{2}\): Polarization levels stay the same in 2011-2012
compared to 2009-2010.}

If hypothesis two is supported, the result would suggest that political
donors helped to create the polarized political environment that we see
today.

In addition, this paper makes a methodological contribution to campaign
finance research by taking a network approach to measuring donor
polarization by using modularity as a measure of polarization. This
method has been used elsewhere in the social sciences to study
congressional polarization (Waugh et al., n.d.; Zhang et al. 2008) and
polarization in social media networks (Guerra et al. 2013; Garcia et al.
2015; Conover et al. 2011). This paper conceives of the political donor
landscape of donors and candidates acting as nodes who are connected by
donations that act as edges. This method is important in studying
political donor networks because it takes into consideration real-world
actions, such as in network studies of polarization among member of
congress where voting records (Guerra et al. 2013) and co-sponsorships
(Zhang et al. 2008) are used to study polarization opposed to surveys
administered to donors that rely on self-reported ideology and
partisanship.

\hypertarget{wisconsin-context}{%
\section{Wisconsin Context}\label{wisconsin-context}}

Both Wisconsin's legislators and mass public are among the most
polarized in the nation (Cramer 2016), and the state been used by
academics to examine how political actions unfold in contentious and
highly divisive environment (Bode et al. 2018). Although many state
legislatures are also experiencing polarization (Shor 2015), Wisconsin
is unique in that there is a single event that many point to in creating
``the most politically divisive place in America'' (Kaufman 2012).

In 2011, newly-elected Republican Governor Scott Walker introduced Act
10, a ``budget reconciliation bill'' that stripped public school
teachers of collective bargaining via their union. Up to 100,000 people
protested this ``anti-union bill'' at the State Capitol and even
occupied the capitol building for a period of time (Sewell 2011).
Democratic lawmakers fled to Illinois in an effort to delay or stop the
bill from passing into law (Layton 2011). In 2012 there was an
unsuccessful election to recall Governor Walker.

Wisconsin Governor Scott Walker's self-anointed ``divide and conquer''
politics (Blake 2012) has left a political divide in Wisconsin that
persists to today. The result is that ``divisive politics ruled
Wisconsin over the last decade'' (Marley and Beck 2019). The Marquette
Law School poll headed by Charles Franklin has called public opinion in
Wisconsin a ``lesson in the two worlds of Wisconsin'' where ``it seems
often as if people have not only differing opinions but differing views
of facts and realities'' (Borsuk 2017).

{[}Why Wisconsin is especially good for studying your question{]}

\hypertarget{methodology}{%
\section{Methodology}\label{methodology}}

All data on political contributions came from the Wisconsin Campaign
Finance Information System (``Wisconsin Campaign Information System,''
n.d.). I exported all contributions to State Assembly, State Senate, and
Gubernatorial races from the 2010, 2012, and 2014 elections. This
dataset does not include donations to party committees, although it does
include disbursements from these committees. I manually created a table
of the parties of each of all the campaigns receiving contributions in
this timeframe and added the party of the campaign receiving the
donation to this dataset.

To clean and analyze my data I used the statistical programming language
R (R Core Team 2013; Wickham et al. 2019). I started with 1,499,603
donations. I then filtered out 3,503 unitemized/ anonymous donations,
removed punctuation from the names of the donors, and used Open Refine
(Kelli 2013) via the \texttt{refinr} R package (Muir 2018) to
standardize names (for example, Jim versus James). Next, I created a
unique identifier for donors by combining their standardized name with
their zip code. This identifier was created to be able to link donors
who contributed across multiple campaigns in multiple years without
considering two different people, with the same name, from different
locations to be the same person.

Next, I estimated the partisanship of each donor in each election cycle
by taking the percent of donations that each donor gave to Republicans
divided by their donations to Republicans and Democrats. I took that
``percent donated to Republicans'' and rescaled it from -1 to 1, where
-1 represents the most Democratic donors, and 1 the most Republican
donors. I also calculated each individual's party bin: if more than 75\%
of donations were to Democrats, they were labeled as a Democrat; if more
than 75\% of donations were to Republicans, they were labeled as a
Republican; if their donations were somewhere inbetween, they were
labeled as being a bipartisan donor.

To quantify the levels of polarization in each election cycle, I
calculated two statistics: network modularity and average absolute
partisanship of donors.

First, political donations can be thought of as a network where donors
and candidates are nodes and donations connecting donors and candidates
are edges. This conceptualization of the political donor landscape as
network allows us to examine the network structure and calculate network
statistics on the graph of donors and candidates. One of the most useful
network statistics for measuring polarization in a network is modularity
(Newman 2006).

The modularity of a graph measures the strength of the division of
groups (such as political parties) by calculating ``the number of edges
falling within groups minus the expected number in an equivalent network
with edges placed at random'' (Newman 2006). The modularity of a network
falls in range {[}-1/2, 1{]}. If the modularity is positive, the number
of edges that remain within each group is greater than the expected
number to remain in-group based on chance. The higher the modularity,
the greater the concentration of edges within each groups. In other
words, the higher the modularity of a network, the higher the
polarization among the groups. Formally, the equation to calculate
modularity Q is:

\[Q = \frac{1}{2m} \sum_{ij}\left[A_{ij} - \frac{k_{i}k_{j}}{2m} \right]\delta(g_{i},g_{j})\]

In this equation \(m = \frac{1}{2}\sum_{i}k_{i}\) is equal to the
strength of all the ties in the network, \(k_{i}=\sum_{j}A_{ij}\) is the
strength/ weighted degree of the \(i\)th node, \(g_{i}\) is the group
(in this case, party/ party bin) to which the \(i\) belong, and
\(\delta(g_{i},g_{j}) = 1\) if \(i\) and \(j\) belong to the same group
(party/ party bin) and 0 if they do not belong to the same party/ party
bin.

I calculated the modularity of the network graphs of each election cycle
(2010, 2012, 2014) using the \texttt{igraph} R package (Csardi and
Nepusz 2006). I used candidates' declared parties and donors' party bin
as the groups for the modularity calculation. The modularity of the
network graph of each election is in Table 1.

In addition to calculating the change in modularity of each of the
election cycles, I also analyzed the change in mean absolute
partisanship of the donors in each election cycle.

I defined a donor's absolute partisanship as the absolute value of their
partisanship score (which is on a scale from -1 to 1). Therefore, the
larger a donor's absolute the partisanship, the higher percentage of
their money that they contributed to a single party. To calculate the
significance in the difference of the mean absolute partisanship, I use
a bootstrap methodology with 1,000 replications using the \texttt{infer}
R package (Bray et al. 2020). This paper uses a non-parametric
permutation method because of the non-Normal distribution of
partisanship of the donors (98\% of donors across all election cycles
only contribute to a single party).

To conduct a hypothesis test with a permutation, you first compute the
mean for each group of the measure that you're interested in--in this
case we are calculating the mean partisanship of donors in two given
election cycles. This is difference
\(d = \overline{X_{1}}-\overline{X_{2}}\). Next, you pool the data and
randomly draw new groups of data of equal sizes of the original groups.
You then calculate the difference in sample means from these random
draws and compare them to your original sample mean (Wilcox 2003). This
paper uses 1,000 replications, primarily to be able to calculate a
p-value to the thousandths place. A p-value for a permutation test is
done by calculating the proportion of randomly drawn \(d's\) that are
greater than the original \(d\) (Butar and Park 2008). In essence, you
figure out what proportion of your randomized draws have a sample mean
that is greater or less than your observed data--how many randomized
groupings having a result that is as or more extreme than your observed
groups. If your p-value is below your pre-specified level (this paper
uses the standard .05) you can reject the null hypothesis.

The results of the bootstrap are found in Table 2.

\hypertarget{results}{%
\section{Results}\label{results}}

The results of this analysis show that political donors in Wisconsin
polarized during the 2012 election cycle, the same time that mass
polarization occurred in the state. This phenomenon is best visualized
in Figure 1. This figure uses the Yifan Hu layout algorithm (Hu 2005) in
the Gephi software (Bastian, Heymann, and Jacomy 2009), a force-directed
graphical layout of networks that seeks to repulse clusters of nodes
from one another. The Yifan Hu layout algorithm is a standard among
social scientists studying networks such as online networks (Rehman et
al. 2020; Adalat, Niazi, and Vasilakos 2018; Khonsari et al. 2010;
Hemsley et al. 2015). This visual representation shows two distinct
clusters of donors (Democrats and Republicans) that are reasonably close
to one another in the 2010 election cycle and then polarize
significantly in the 2012 election cycle and remain polarized in 2014.

This graphical representation reflects statistical measures of
polarization within the networks. Table 1 show the modularity of the
networks in the 2010, 2012, and 2014 election cycles. In 2010, the
modularity of the donor network is 0.4. The modularity for the 2012
cycle climbs to 0.49 and settles in at 0.48 during the 2014 cycle. The
interpretation of modularity is the higher the number, the more observed
polarization within the network. As such, the rise in modularity in the
2012 cycle depicts polarization within the donor network in 2012
compared to 2010. And then the steady modularity in the 2014 cycle
reveals a stabilization of the level of polarization observed in the
2012 election cycle.

One limitation of a modularity calculation is that it does not quantify
uncertainty. To validate the results of the modularity calculation, I
conducted a hypothesis test. Table 2 compares the average absolute donor
partisanship in 2012 compared to 2010 and 2014 compared to 2012. As the
table shows, donors in the 2012 election cycle became much more partisan
with an average change of absolute of partisanship of 0.04212 (CI =
0.04025-0.04405, p-value = \textless.001). However, there was not a
statistically signiciant change in mean absolute partisanship in the
2014 election cycle compared to the 2012 election cycle
(\ensuremath{-2.8\times 10^{-4}}, CI = -0.00089-0.00036, p-value =
0.376).

Taken together, the results of the modularity calculations and the
hypothesis tests support the rejection of the null of \(H_{1}\) and fail
to reject the null of the alternative \(H_{2}\). In other words,
political donors in Wisconsin had a statistically significant increase
in polarization in the 2012 election cycle--the same time as when other
scholars and experts point towards the mass polarization of the state.
{[}Find mass polarization data, hopefully MU law poll{]}

Additional data and graphs are provided to contextualize the
polarization that is observed among political donors in Wisconsin.
Figure 2 shows the partisan flow of political donors across the election
cycles, including the massive influx of new donors in both the 2012 and
2014 election cycles. Figure 3 shows the partisan shift of donors who
contributed in both the 2010 and 2012 election cycles. Figure 4 shows
the geographical shifts of donor partisanship.

\hypertarget{discussion}{%
\section{Discussion}\label{discussion}}

The failure to reject the null of \(H_{1}\) suggests that political
donors were likely not the main contributors to the extreme levels of
polarization first seen in the state in 2012. Other factors such as
Governor Scott Walker's Act 10 and a more polarized primary electorate
in the wake of the Tea Party in 2010 (Jacobson 2012), and electing
election Governor Walker in the first place, appear to be the
contributors to mass polarization and political donors in Wisconsin.

These results also provide evidence for the `consumption' model of
political donations. Ansolabehere, de Figueiredo and Syder's (2003)
conclusion that political donations are similar to voting in that they
are both acts of political consumption are borne out in the results of
this paper. Polarization of political donors happened in unison with the
polarization of the electorate. The conclusion that we can draw is that
the polarization of these two groups of people were a behavioral,
participatory response to a changing political environment. Both the
electorate and donors have specific acts of political consumption
(voting and donating, respectively) that were both impacted in the same
way at the same time.

Further evidence for this consumption model is the idea that political
donations are an extension of voting in the broader realm of political
participation. The inflow of new donors show in Figure 2 suggests that
the same mechanism that triggered mass polarization also spurred members
of the mass electorate to go beyond voting and make political
contributions. Previous research by Oklobzija (2016) reached a similar
conclusion that ``politically polarizing events bear dividends for
extremist lawmakers'' in California who raised more money as a result of
polarizing political events. Even though new donors are the likely
explanation for most of the polarization observed in the networks,
donors that contributed in both the 2010 and 2012 election cycles also
showed significant movement. The shift among donors whose contributions
are not purely partisan shift to be more Republican. Figure 3 shows how
donors who were not pure partisan donors in 2010 much more often became
purely Republican donors compared to Democartic donors. {[}Add
split-ticket voting discussion{]}

{[}Add paragraph on geographic sorting{]}

In short, it appears that political donations are an extension of
voting, an outlet for political participation when individuals perceive
the stakes of the election to be high (Hill and Huber 2017). And so it
would be reasonable to find that political donors are not the cause of
political polarization. But in fact, more polarized donors are a
reflection of polariztion seen elsewhere in American politics.

\newpage

\hypertarget{tables}{%
\section{Tables}\label{tables}}

\begin{longtable}[]{@{}lr@{}}
\caption{Modularity calculation for the donor networks in each election
cycle. Higher modularity means more polarization.}\tabularnewline
\toprule
Election Cycle & Modularity\tabularnewline
\midrule
\endfirsthead
\toprule
Election Cycle & Modularity\tabularnewline
\midrule
\endhead
2010 & 0.3987172\tabularnewline
2012 & 0.4912366\tabularnewline
2014 & 0.4797903\tabularnewline
\bottomrule
\end{longtable}

\newpage

\begin{longtable}[]{@{}lrll@{}}
\caption{Bootstrapped difference-in-means test with 1,000 replications
comparing mean partisanship of donors.}\tabularnewline
\toprule
Election Cycle & T & CI & p\tabularnewline
\midrule
\endfirsthead
\toprule
Election Cycle & T & CI & p\tabularnewline
\midrule
\endhead
2012 compared to 2010 & 0.04212 & 0.04025-0.04405 &
\textless.001\tabularnewline
2014 compared to 2012 & -0.00028 & -0.00089-0.00036 &
0.376\tabularnewline
\bottomrule
\end{longtable}

\newpage

\hypertarget{figures}{%
\section{Figures}\label{figures}}

\hypertarget{figure-1}{%
\subsubsection{Figure 1}\label{figure-1}}

\begin{figure}

{\centering \includegraphics[width=0.73\linewidth]{../figures/fig1} 

}

\caption{Visual representation of Wisconsin donor networks in the 2010, 2012 and 2014 election cycle using the Yifan Hu layout algorithm. Each dot/ node is a donor or campaign and lines/ edges connecting them are donations. Nodes sized by in-degree (incoming donations. Nodes and edges are colored by the partisanship of the donor. Percentages on the bars reprsent the percent of donors in each party bin.}\label{fig:unnamed-chunk-10}
\end{figure}

\newpage

\hypertarget{figure-2}{%
\subsubsection{Figure 2}\label{figure-2}}

\begin{figure}
\includegraphics[width=1\linewidth]{../figures/fig2} \caption{Sankey diagram of the flow of political donors in 2010, 2012, and 2014 election cycles in Wisconsin. The vertical bars are proportional to the number of donors in each bin.}\label{fig:unnamed-chunk-11}
\end{figure}

\newpage

\hypertarget{figure-3}{%
\subsubsection{Figure 3}\label{figure-3}}

\begin{figure}
\includegraphics[width=1\linewidth]{../figures/fig3} \caption{Every dot is a donor who contributed in 2010 and 2012. The bigger the dot, the more money they contribted. The x-axis is their partisanship in the 2010 election cycle and the y-axis is their partisanship in the 2012 election cycle. If the donor is to the right of the center diagonal line, they became more Republican. If they are to the left of the line, they became more Democratic.}\label{fig:unnamed-chunk-12}
\end{figure}

\newpage

\hypertarget{figure-4}{%
\subsubsection{Figure 4}\label{figure-4}}

\begin{figure}
\includegraphics[width=0.9\linewidth]{../figures/fig4} \caption{This box and whisker plot is grouped by the partisanship of the donors in the 2010 and 2012 election cycles. Note that the y-axis is shown on a log10 scale for clarity. The partisan distribution is shown along the bottom of the x-axis.}\label{fig:unnamed-chunk-13}
\end{figure}

\newpage

\hypertarget{figure-5}{%
\subsubsection{Figure 5}\label{figure-5}}

\begin{figure}
\includegraphics[width=0.9\linewidth]{../figures/fig5} \caption{This map shows the polarization of donor networks across Wisconsin's counties based on the net change of donors in each county per 10,000 residents. The red counties had a net increase in Republican donors, blue counties had a net increase for Democrats, and the purple counties had little or no change.}\label{fig:unnamed-chunk-14}
\end{figure}

\newpage

\hypertarget{references}{%
\section*{References}\label{references}}
\addcontentsline{toc}{section}{References}

\hypertarget{refs}{}
\leavevmode\hypertarget{ref-adalat2018}{}%
Adalat, Mohsin, Muaz A. Niazi, and Athanasios V. Vasilakos. 2018.
``Variations in Power of Opinion Leaders in Online Communication
Networks.'' \emph{Royal Society Open Science} 5 (10).

\leavevmode\hypertarget{ref-akey2015}{}%
Akey, Pat. 2015. ``Valuing Changes in Political Networks: Evidence from
Campaign Contributions to Close Congressional Elections.'' \emph{The
Review of Financial Studies} 28 (11): 3188--3223.

\leavevmode\hypertarget{ref-ansolabehere2003}{}%
Ansolabehere, Stephen, John M. de Figueiredo, and James M. Snyder Jr.
2003. ``Why Is There so Little Money in U.s. Politics.'' \emph{Journal
of Economic Perspectives} 17 (1): 105--30.

\leavevmode\hypertarget{ref-barber2016b}{}%
Barber, Michael J., Daniel M. Butler, and Jessica Preece. 2016. ``Gender
Inequalities in Campaign Finance.'' \emph{Quarterly Journal of Political
Science} 1 (2): 219--48.

\leavevmode\hypertarget{ref-barber2016c}{}%
Barber, Michael J., Brandice Canes-Wrone, and Sharece Thrower. 2016.
``Ideologically Sophisticated Donors: Which Candidates Do Individual
Contributors Finance.'' \emph{American Journal of Political Science} 61
(2): 1057--72.

\leavevmode\hypertarget{ref-gephi}{}%
Bastian, Mathieu, Sebastien Heymann, and Mathieu Jacomy. 2009. ``Gephi:
An Open Source Software for Exploring and Manipulating Networks.''
\url{http://www.aaai.org/ocs/index.php/ICWSM/09/paper/view/154}.

\leavevmode\hypertarget{ref-blake2012}{}%
Blake, Aaron. 2012. ``Scott Walker Said Budget Strategy in Wisconsin Was
'Divide and Conquer'.'' \emph{The Washington Post}, May.

\leavevmode\hypertarget{ref-bode2018}{}%
Bode, Leticia, Stephanie Edgerly, Chris Wells, Itay Gabay, Charles
Franklin, Lew Friedland, and Dhavan V. Shah. 2018. ``Participation in
Contentious Politics: Rethinking the Roles of News, Social Media, and
Conversation Amid Divisiveness.'' \emph{Journal of Information
Technology \& Politics} 15 (3): 215--29.

\leavevmode\hypertarget{ref-bonica2017}{}%
Bonica, Adam. 2017. ``Professional Networks, Early Fundraising, and
Electoral Success.'' \emph{Election Law Journal: Rules, Politics, and
Policy} 16 (1): 153--71.

\leavevmode\hypertarget{ref-bonneau2007}{}%
Bonneau, Chris W. 2007. ``Campaign Fundraising in State Supreme Court
Elections.'' \emph{Social Science Quartlery} 88 (1): 68--85.

\leavevmode\hypertarget{ref-borsuk2017}{}%
Borsuk, Alan. 2017. ``New Poll Gives Vivid Look into Polarized Political
Perceptions.'' June 29, 2017.
\url{https://law.marquette.edu/poll/2017/06/29/new-poll-gives-vivid-look-into-polarized-political-perceptions/}.

\leavevmode\hypertarget{ref-bowler2015}{}%
Bowler, Shaun, and Todd Donovan. 2015. ``Campaign Money, Congress, and
Perceptions of Corruption.'' \emph{American Politics Research} 44 (2):
272--95.

\leavevmode\hypertarget{ref-bramlett2011}{}%
Bramlett, Brittany H., James G. Gimpel, and Frances E. Lee. 2011. ``The
Political Ecology of Opinion in Big-Donor Neighborhoods.''
\emph{Political Behavior} 33: 565--600.

\leavevmode\hypertarget{ref-infer}{}%
Bray, Andrew, Chester Ismay, Evgeni Chasnovski, Ben Baumer, Mine
Cetinkaya-Rundel, Simon Couch, Ted Laderas, et al. 2020. \emph{Infer:
Tidy Statistical Inference}.
\url{https://cran.r-project.org/web/packages/infer/index.html}.

\leavevmode\hypertarget{ref-butar2008}{}%
Butar, Ferry, and Jae-Wan Park. 2008. ``Permutation Tests for Comparing
Two Populations.'' \emph{MSME} 3 (September): 19--30.

\leavevmode\hypertarget{ref-conover2011}{}%
Conover, Michael D., Jacob Ratkiewicz, M. Francisco, B. Gonçalves, F.
Menczer, and A. Flammini. 2011. ``Political Polarization on Twitter.''
In \emph{ICWSM}.

\leavevmode\hypertarget{ref-cooper2010}{}%
Cooper, Michael J., Huseyin Gulen, and Alexei V. Ovtchinnikov. 2010.
``Corporate Political Contributions and Stock Returns.'' \emph{The
Journal of Finance} 65 (2): 687--724.

\leavevmode\hypertarget{ref-cramer2016}{}%
Cramer, K. J. 2016. \emph{The Politics of Resentment: Rural
Consciousness in Wisconsin and the Rise of Scott Walker}. Chicago
Studies in American Politics. University of Chicago Press.
\url{https://books.google.com/books?id=Rg2ZCwAAQBAJ}.

\leavevmode\hypertarget{ref-crowder-meyer2018}{}%
Crowder-Meyer, Melody, and Rosalyn Cooperman. 2018. ``Can't Buy Them
Love: How Party Culture Among Donors Contributes to the Party Gap in
Women's Representation.'' \emph{The Journal of Politics} 80 (4):
1211--24.

\leavevmode\hypertarget{ref-igraph}{}%
Csardi, Gabor, and Tamas Nepusz. 2006. ``The Igraph Software Package for
Complex Network Research.'' \emph{InterJournal} Complex Systems: 1695.
\url{http://igraph.org}.

\leavevmode\hypertarget{ref-ellis2017}{}%
Ellis, William Curtis, Joseph T. Ripberger, and Colin Swearingen. 2017.
``Public Attention and Head-to-Head Campaign Fundraising: An Examination
of U.s. Senate Elections.'' \emph{American Review of Politics} 36 (1):
30--53.

\leavevmode\hypertarget{ref-garro2020}{}%
Fowler, Anthony, Haritz Garro, and Jörg L. Spenkuch. 2020. ``Quid Pro
Quo? Corporate Returns to Campaign Contributions.'' \emph{The Journal of
Politics} 82 (3): 844--58.

\leavevmode\hypertarget{ref-francia2003}{}%
Francia, Peter L., John C. Green, Paul S. Herrnson, Lynda W. Powell, and
and Clyde Wilcox. 2003. \emph{The Financiers of Congressional
Elections}. New York, NY: Columbia University Press.

\leavevmode\hypertarget{ref-francia2005}{}%
Francia, Peter L., John C. Green, Paul S. Herrnson, Lynda W. Powell, and
Clyde Wilcox. 2005. ``Limousine Liberals and Corporate Conservatives:
The Financial Constituencies of the Democratic and Republican Parties.''
\emph{Social Science Quarterly} 86 (4): 761--78.

\leavevmode\hypertarget{ref-garcia2015}{}%
Garcia, David, Adiya Abisheva, Simon Schweighofer, Uwe Serdült, and
Frank Schweitzer. 2015. ``Ideological and Temporal Components of Network
Polarization in Online Political Participatory Media.'' \emph{Policy \&
Internet} 7 (1): 46--79.

\leavevmode\hypertarget{ref-gordon2007}{}%
Gordon, Sanford C., Catherine Hafer, and Dimitri Landa. 2007.
``Consumption or Investment? On Motivations for Political Giving.''
\emph{The Journal of Politics} 69 (4).

\leavevmode\hypertarget{ref-guerra2013}{}%
Guerra, P. H. Calais, Wagner Meira Jr., Clair Cardie, and R. Kleinberg.
2013. ``Party Polarization in Congress: A Network Science Approach.''
\emph{Proceedings of the 7th International Conference on Weblogs and
Social Media, ICWSM 2013}, January, 215--24.

\leavevmode\hypertarget{ref-harden2016}{}%
Harden, Jeffrey J., and Justin H. Kirkland. 2016. ``Do Campaign Donors
Influence Polarization? Evidence from Public Financing in the American
States.'' \emph{Legislative Studies Quarterly} 41 (1): 119--1542.

\leavevmode\hypertarget{ref-hemsley2015}{}%
Hemsley, Bronwyn, Stephen Dann, Stuart Palmer, Meredith Allan, and Susan
Balandin. 2015. ```We Definitely Need an Audience': Experiences of
Twitter, Twitter Networks and Tweet Content in Adults with Severe
Communication Disabilities Who Use Augmentative and Alternative
Communication (Aac).'' \emph{Disability and Rehabilitation} 37 (17):
1531--42. \url{https://doi.org/10.3109/09638288.2015.1045990}.

\leavevmode\hypertarget{ref-hill2017}{}%
Hill, Seth J., and Gregory A. Huber. 2017. ``Representativeness and
Motivations of the Contemporary Donorate: Results from Merged Survey and
Administrative Records.'' \emph{Political Behavior} 39 (March): 3--29.

\leavevmode\hypertarget{ref-yifanhu}{}%
Hu, Yifan. 2005. ``Efficient, High-Quality Force-Directed Graph
Drawing.'' \emph{Mathematica Journal} 10 (1): 37--71.

\leavevmode\hypertarget{ref-jacobson2012}{}%
Jacobson, Gary C. 2012. ``The Electoral Origins of Polarized Politics:
Evidence from the 2010 Cooperative Congressional Election Study.''
\emph{American Behavioral Scientist} 56 (12): 1612--30.

\leavevmode\hypertarget{ref-kaufman2012}{}%
Kaufman, Dan. 2012. ``How Did Wisconsin Become the Most Politically
Divisive Place in America?'' \emph{The New York Times Magazine}, May.

\leavevmode\hypertarget{ref-keena2019}{}%
Keena, Alex, and Misty Knight-Finley. 2019. ``Are Small Donors
Polarizing? A Longitudinal Study of the Senate.'' \emph{Election Law
Journal: Rules, Politics, and Policy} 18 (2): 132--44.

\leavevmode\hypertarget{ref-openrefine}{}%
Kelli, Ham. 2013. ``OpenReinfe (Version 2.5).'' \emph{Journal of the
Medical Library Association} 101 (3): 233--34.

\leavevmode\hypertarget{ref-khonsari2010}{}%
Khonsari, K. K., Z. A. Nayeri, A. Fathalian, and L. Fathalian. 2010.
``Social Network Analysis of Iran's Green Movement Opposition Groups
Using Twitter.'' In \emph{2010 International Conference on Advances in
Social Networks Analysis and Mining}, 414--15.

\leavevmode\hypertarget{ref-kitchens2016}{}%
Kitchens, Karin E., and Michele L. Swers. 2016. ``Why Aren't There More
Republican Women in Congress? Gender, Partisanship, and Fundraising
Support in the 2010 and 2012 Elections.'' \emph{Politics \& Gender} 12
(4): 648--76.

\leavevmode\hypertarget{ref-laurison2016}{}%
Laurison, Daniel. 2016. ``Social Class and Political Engagement in the
United States.'' \emph{Sociology Compass} 10 (9): 684--97.

\leavevmode\hypertarget{ref-layton2011}{}%
Layton, Lyndsey. 2011. ``'Wisconsin 14' Group of Democratic Senators
Returns, Greeted by Thousands at Capitol.'' \emph{The Washington Post},
March.

\leavevmode\hypertarget{ref-marley2019}{}%
Marley, Patrick, and Molly Beck. 2019. ``Divisive Politics Ruled
Wisconsin over the Last Decade.'' \emph{Milwaukee Journal Sentinel},
December.

\leavevmode\hypertarget{ref-mccarty2006}{}%
McCarty, Nolan, Keith T. Poole, and Howard Rosenthal. 2006.
\emph{Polarizaed America: The Dancedance of Ideology and Unequal
Riches}. Cambridge, Mass: MIT Press.

\leavevmode\hypertarget{ref-refinr}{}%
Muir, Chris. 2018. \emph{Refinr: Cluster and Merge Similar Values Within
a Character Vector}.
\url{https://cran.r-project.org/web/packages/refinr/index.html}.

\leavevmode\hypertarget{ref-newman2006}{}%
Newman, M. E. J. 2006. ``Modularity and Community Structure in
Networks.'' \emph{Proceedings of the National Academy of Sciences} 103
(23): 8577--82. \url{https://doi.org/10.1073/pnas.0601602103}.

\leavevmode\hypertarget{ref-oklobzija}{}%
Oklobdzija, Stan. 2016. ``Closing down and Cashing in: Extremism and
Political Fundraising.'' \emph{State Politics \& Policy Quarterly} 17
(2): 201--24.

\leavevmode\hypertarget{ref-palmer2008}{}%
Palmer, Vernon Valentine, and John Levendis. 2008. ``The Louisiana
Supreme Court in Question: An Empirical and Statistical Study of the
Effects of Money on the Judicial Function.'' \emph{Tulane Law Review} 82
(4): 1291--1314.

\leavevmode\hypertarget{ref-pew2017}{}%
Pew Research Center. 2017. ``The Partisan Divide on Political Values
Grows Even Wider.'' online.

\leavevmode\hypertarget{ref-laraja2011}{}%
Raja, Raymond J. La, and David L. Wiltse. 2012. ``Don't Blame Donors for
Ideological Polarization of Political Parties: Ideological Change and
Stability Among Political Contributors, 1972-2008.'' \emph{American
Politics Research} 40 (3): 501--30.

\leavevmode\hypertarget{ref-r}{}%
R Core Team. 2013. \emph{R: A Language and Environment for Statistical
Computing}. Vienna, Austria: R Foundation for Statistical Computing.
\url{http://www.R-project.org/}.

\leavevmode\hypertarget{ref-rehman2020}{}%
Rehman, Ateeq Ur, Aimin Jiang, Abdul Rehman, Anand Paul, Sadia din, and
Muhammad Tariq Sadiq. 2020. ``Identification and Role of Opinion Leaders
in Information Diffusion for Online Discussion Network.'' \emph{Journal
of Ambient Intelligence and Humanized Computing}, January.

\leavevmode\hypertarget{ref-roscoe2005}{}%
Roscoe, Douglas D., and Shannon Jenkins. 2005. ``A Meta-Analysis of
Campaign Contributions' Impact on Roll Call Voting.'' \emph{Social
Science Quartlery} 86 (1): 52--68.

\leavevmode\hypertarget{ref-sewell2011}{}%
Sewell, Abby. 2011. ``Protesters Out in Force Nationwide to Oppose
Wisconsin's Anti-Union Bill.'' \emph{Los Angeles Times}, February.

\leavevmode\hypertarget{ref-shor2015}{}%
Shor, Boris. 2015. ``Polarization in American State Legislatures.'' In
\emph{American Gridlock: The Sources, Character, and Impact of Political
Polarization}, edited by James A. Thurber and AntoineEditors Yoshinaka,
203--21. Cambridge University Press.
\url{https://doi.org/10.1017/CBO9781316287002.011}.

\leavevmode\hypertarget{ref-stratmann1991}{}%
Stratmann, Thomas. 1991. ``What Do Campaign Contributions Buy?
Deciphering Causal Effects of Money and Votes.'' \emph{Southern Economic
Journal} 57 (3): 606--20.

\leavevmode\hypertarget{ref-thomsen2017}{}%
Thomsen, Danielle M., and Michele L. Swers. 2017. ``Which Women Can Run?
Gender, Partisanship, and Candidate Donor Networks.'' \emph{Political
Research Quarterly} 70 (2): 449--63.

\leavevmode\hypertarget{ref-torres-spelliscy2017}{}%
Torres-Spelliscy, Ciara. 2017. ``Time Suck: How the Fundraising
Treadmill Diminishes Effetive Governance.'' \emph{Seton Hall Legislative
Journal} 42 (December).

\leavevmode\hypertarget{ref-waugh2009}{}%
Waugh, Andrew Scott, Liuyi Pei, James H. Fowler, Peter J. Mucha, and
Mason Alexander Porter. n.d. ``Party Polarization in Congress: A Network
Science Approach.''

\leavevmode\hypertarget{ref-tidyverse}{}%
Wickham, Hadley, Mara Averick, Jennifer Bryan, Winston Chang, Lucy
D'Agostino McGowan, Romain François, Garrett Grolemund, et al. 2019.
``Welcome to the tidyverse.'' \emph{Journal of Open Source Software} 4
(43): 1686. \url{https://doi.org/10.21105/joss.01686}.

\leavevmode\hypertarget{ref-wilcox2003}{}%
Wilcox, Rand R. 2003. ``8 - Comparing Two Independent Groups.'' In
\emph{Applying Contemporary Statistical Techniques}, edited by Rand R.
Wilcox, 237--84. Burlington: Academic Press.
\url{https://doi.org/https://doi.org/10.1016/B978-012751541-0/50029-8}.

\leavevmode\hypertarget{ref-cfis}{}%
``Wisconsin Campaign Information System.'' n.d.
\url{https://cfis.wi.gov/\#}.

\leavevmode\hypertarget{ref-zhang2008}{}%
Zhang, Yan, A. J. Friend, Amanda L. Traud, Mason A. Porter, James H.
Fowler, and Peter J. Mucha. 2008. ``Community Sructure in Congressional
Cosponsorship Networks.'' \emph{Physica A: Statistical MEchanics and Its
Applications} 387 (1): 1705--12.





\newpage
\singlespacing 
\end{document}
